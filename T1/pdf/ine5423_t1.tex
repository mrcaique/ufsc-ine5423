\documentclass[fleqn]{article}
\usepackage[utf8]{inputenc}
\usepackage{graphicx} % inclusão de figuras
\graphicspath{ {imgs/} } % local padrão das imagens usadas neste documento
\usepackage{fancyvrb} % Inclusão de comandos na área de verbatim
\usepackage{caption} % Permite a remoção do label "Figura 1"
\usepackage[brazil]{babel} % Texto em português do Brasil
\usepackage[a4paper, left=20mm, right=20mm, top=20mm, bottom=20mm]{geometry}
% Formatação da página
\usepackage{amsmath} % Símbolos matemáticos
\usepackage{enumerate} % Lista com índice em números romanos
\usepackage{listings} % Bloco lstlisting
\usepackage[colorlinks, urlcolor=blue, citecolor=red]{hyperref}
% Uso de hyperlinks

\title{\textbf{INE5423 - Banco de Dados I (UFSC - 2016/1)}}
\author{
    Caique Rodrigues Marques \\
    {\texttt{c.r.marques@grad.ufsc.br}}
    \thanks{Todos os códigos relacionados a este projeto podem ser encontrados
    \href{https://github.com/mrcaique/ufsc-ine5423}{neste repositório}.}
    \and
    Lucas Ribeiro Neis \\
    {\texttt{lucasneis@hotmail.com.br}}
    \vspace{-5mm}
}
\date{}

\begin{document}

\maketitle

\section*{Questão 1}
    \begin{figure*}[h!]
        \centering
        \includegraphics[scale=0.6]{ine5423_diagram1.png}
        \caption*{\textit{A herança definida em "Avaliação" é total e exclusiva}}
        %\label{fig:my_label}
    \end{figure*}
    
\section*{Questão 2}
    \begin{Verbatim}[commandchars=+\[\]]
    Prova(+underline[número], +underline[Ncartão], nome, nota, data)
        Ncartão referencia Aluno(nCartão)
    Trabalho(+underline[número], nome, tipo_entrega, de, até)
    Grupo(+underline[número], nome)
    Aluno(+underline[nCartão], nome, sexo)
    T_G(+underline[nGrupo], +underline[nTrab], nota)
        nGrupo referencia Grupo(número)
        nTrab referencia Trabalho(número)
    G_A(+underline[nAluno], +underline[nGrupo])
        nAluno referencia Aluno(nCartão)
        nGrupo referencia Grupo(número)
    \end{Verbatim}

\newpage
\section*{Questão 4}
    \begin{enumerate}[I] % Enumerar usando números romanos
        \item Nome do Projeto e Ano de término dos projetos que são
        sequencia de outros e que começaram entre 2000 e 2010.
            \begin{lstlisting}[mathescape]
    $\pi$<NomeProj, AnoFim>(
        $\sigma$<CodProjAnt != NULL $\land$
            AnoInicio $\leq$ '2000' $\land$
            AnoInicio $\geq$ '2010'>(Projeto)
    )
            \end{lstlisting}
        
        \item Código e nome do curso, nome e email das pessoas onde a pessoa tem
        email \texttt{@uni} e o nome do curso que ela cursa é Computação ou a
        pessoa é do sexo Feminino
        \begin{enumerate}[(a)] % Listar usando letras minúsculas
            \item Com operador JOIN.
                \begin{lstlisting}[mathescape]
    $\pi$<CodCurso, NomeCurso, NomePess, EmailPref>(
        $\sigma$<Pessoa.EmailPref LIKE '%@uni%' $\land$(
            Curso.NomeCurso = 'Computacao' $\lor$
            Pessoa.Sexo = 'F')>(
                Pessoa $\bowtie$ Curso
            )
        )
    )
                \end{lstlisting}

            \item Com operador de produto cartesiano.
                \begin{lstlisting}[mathescape]
    $\pi$<CodCurso, NomeCurso, NomePess, EmailPref>(
        $\sigma$<Pessoa.EmailPref LIKE '%@uni%' $\land$
            (
                Curso.NomeCurso = 'Computacao' $\lor$
                Pessoa.Sexo = 'F'
            ) $\land$ Pessoa.CodCurso = Curso.CodCurso>
            (
                ($\pi$<NomePess, EMailPref, Sexo, CodCurso>(Pessoa)) $\times$
                ($\pi$<CodCurso, NomeCurso>(Curso))
            )
    )
                \end{lstlisting}
        \end{enumerate}
        
        \item Nome das pessoas, nome dos projetos, papel das pessoas no projeto
        e o número de pessoas envolvidas no projeto onde o projeto tenha mais
        de um envolvido, seja uma continuação de um projeto anterior e tenha
        sido iniciado após 2000.
            \begin{lstlisting}[mathescape]
    $\pi$<Pessoa.NomePess, Projeto.NomeProj, ProjetoPessoa.PapelPessProj>(
        $\sigma$<count > 1 $\land$ ProjetoPessoa.CodProjAnte != NULL $\land$ AnoInicio > '2000'>
        (Pessoa $\bowtie$ (Projeto $\bowtie$ ProjetoPessoa)) $\bowtie$ (    
            $\pi$<Projeto.CodProj, $\rho$<G(count(NumeroCartao))/count>(Pessoa)>(
                Pessoa $\bowtie$ (Projeto $\bowtie$ ProjetoPessoa)
            )
        )
    )
            \end{lstlisting}
        
        \item Nome das mulheres e seus respectivos cursos onde ela tem um email
        alternativo hospedado fora de \texttt{@uni} e o número de email
        alternativos respeitando a cláusula.
            \begin{lstlisting}[mathescape]
    $\pi$<Pessoa.NomePess, Curso.NomeCurso, G(count(CodCurso))>(
        $\sigma$<Pessoa.sexo = 'F' $\land$ Email NOT LIKE '%@uni%'>(
            Curso $\bowtie$ (Pessoa $\bowtie$ OutroEmail)
        )
    )
            \end{lstlisting}
    
    \newpage
        \item O numero do cartão, nome da pessoa e número de projetos dos ALUNOS
        que LIDERAM dois ou mais projetos
            \begin{lstlisting}[mathescape]
    $\pi$<Pessoa.NumeroCartao, Pessoa.NomePess, G(count(numeroCartao))>(
        $\sigma$<PapelPessProj = 'Lider' $\land$ Curso != NULL>(
            Curso $\bowtie$ (Pessoa $\bowtie$ ProjetoPessoa)
        )
    )
            \end{lstlisting}
    \end{enumerate}
\end{document}
