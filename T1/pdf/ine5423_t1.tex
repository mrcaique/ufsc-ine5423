\documentclass{article}
\usepackage[utf8]{inputenc}
\usepackage{graphicx} % inclusão de figuras
\graphicspath{ {imgs/} } % local padrão das imagens usadas neste documento
\usepackage{fancyvrb} % Inclusão de comandos na área de verbatim
\usepackage{caption} % Permite a remoção do label "Figura 1"
\usepackage[brazil]{babel} % Texto em português do Brasil
\usepackage[a4paper, left=20mm, right=20mm, top=20mm, bottom=20mm]{geometry} % Formatação da página

\title{\textbf{INE5423 - Banco de Dados I (UFSC - 2016/1)}}
\author{
    Caique Rodrigues Marques \\
    {\texttt{c.r.marques@grad.ufsc.br}}
    \and
    Lucas Ribeiro Neis \\
    {\texttt{lucasneis@hotmail.com.br}}
    \vspace{-5mm}
}
\date{}

\begin{document}

\maketitle

\section*{Questão 1}
    \begin{figure*}[h!]
        \centering
        \includegraphics[scale=0.6]{ine5423_diagram1.png}
        \caption*{\textit{A herança definida em "Avaliação" é total e exclusiva}}
        %\label{fig:my_label}
    \end{figure*}
    
\section*{Questão 2}
    \begin{Verbatim}[commandchars=+\[\]]
    Prova(+underline[número], +underline[Ncartão], nome, nota, data)
        Ncartão referencia Aluno(Ncartão)
    Trabalho(+underline[número], nome, tipo_entrega, de, até)
    Grupo(+underline[número], nome)
    Aluno(+underline[nCartão], nome, sexo)
    T_G(+underline[nGrupo], +underline[nTrab], nota)
        nGrupo referencia Grupo(nGrupo)
        nTrab referencia Trabalho(nTrab)
    G_A(+underline[nAluno], +underline[nGrupo])
        nAluno referencia Aluno(nAluno)
        nGrupo referencia Grupo(nGrupo)
    \end{Verbatim}
\end{document}
